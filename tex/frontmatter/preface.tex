\section*{Preface}
This book is divided into the following sections.

%\cref{part:prelim} covers \vocab{preliminary topics}, which are crucial stepping stones for subsequent topics. This includes logic and methods of proofs in \cref{chap:logic-proofs}, and basic set theory in \cref{chap:set-theory}.

%\cref{part:linear-algebra} covers \vocab{linear algebra}, which follows \cite{axler}. \cref{chap:vector-spaces} gives an introduction to vector spaces and subspaces. \cref{chap:finite-dim-vector-spaces} gives an overview of span, linear independence, bases and dimension. \cref{chap:linear-maps} goes through linear maps, kernel and image, matrices, invertibility and isomorphism, as well as products and quotients of vector spaces.

%\cref{part:abstract-algebra} covers \vocab{abstract algebra}, which follows \cite{dummit-foote}. \cref{chap:group-theory} covers group theory.

%\cref{part:real-analysis} covers \vocab{real analysis}, which follows \cite{rudin,apostol}. 

%\cref{part:complex-analysis} covers \vocab{complex analysis}, which follows \cite{ahlfors,lang}. Complex analysis can be viewed as an extension of real analysis, with various overlapping concepts.

%\cref{part:topology} covers \vocab{topology}, which follows \cite{munkres}.

The chapters in this book are structured in the following typical manner. Each chapter begins with a \textbf{theoretical portion}, which starts off with a couple of definitions, followed by theorems, lemmas and propositions built upon the definitions. Each chapter is ended by some \textbf{exercises} accompanied by solutions to them.

The reader is not assumed to have any mathematical prerequisites.